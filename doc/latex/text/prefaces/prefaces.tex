\begin{center}
{\LARGE\bfseries\titulo}\\
\end{center}
\begin{center}
\autor\
\end{center}

\section*{Resumen}

\bigskip

Las redes neuronales son una herramienta muy potente que pueden ser entrenadas para resolver todo tipo de problemas.

\bigskip

En esta práctica se trabajará con ellas para reconocer caracteres escritos a mano. Haciendo uso de la base de datos MNIST de números manuscritos y del framework de desarrollo Keras, se configurará y entrenará una red neuronal para que, al evaluarla, reconozca el mayor número de dígitos posibles.

\bigskip

\noindent{\textbf{Palabras clave}: \textit{\keywords}\\

\bigskip

El formato de la documentación de este trabajo ha sido basado en la plantilla \LaTeX\space de \href{https://github.com/erseco}{https://github.com/erseco}.
